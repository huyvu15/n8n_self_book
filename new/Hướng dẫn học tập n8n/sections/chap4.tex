\chapter{Làm việc với dữ liệu trong n8n}

\section{Biến, JSON, Expressions}

\subsection{Biến trong n8n: Cách lưu trữ và sử dụng thông tin}

Trong cuộc sống hàng ngày, chúng ta thường ghi chú thông tin để sử dụng sau này. Trong n8n, ``biến'' chính là những ghi chú kỹ thuật số.

\subsubsection{Ví dụ: Tự động gửi tin nhắn chào mừng}

Giả sử bạn muốn chào mừng khách hàng mới đăng ký:

\begin{enumerate}
    \item Tạo biến workflow để lưu lời chào:
    \begin{verbatim}
    welcomeMessage: Chào mừng bạn đến với dịch vụ của chúng tôi!
    companyName: Công ty ABC
    \end{verbatim}

    \item Khi có người đăng ký, sử dụng biến trong email:
    \begin{verbatim}
    Tiêu đề: {{ $vars.welcomeMessage }}
    Nội dung: Cảm ơn bạn đã đăng ký dịch vụ của {{ $vars.companyName }}
    \end{verbatim}
\end{enumerate}

% \begin{figure}[h]
%     \centering
%     \includegraphics[width=0.8\textwidth]{images/bien-trong-workflow-email}
%     \caption{Biến trong workflow thực tế}
%     \label{fig:bien-workflow}
% \end{figure}

\textbf{Cách thiết lập biến đơn giản:}
\begin{enumerate}
    \item Mở workflow
    \item Nhấp vào biểu tượng bánh răng
    \item Chọn ``Variables''
    \item Thêm tên và giá trị cho biến
\end{enumerate}

\subsection{JSON: Ngôn ngữ dữ liệu của n8n}

JSON giống như một bảng thông tin có tổ chức. Ví dụ, thông tin người dùng trong JSON:

\begin{verbatim}
{
  "hoTen": "Nguyễn Văn A",
  "email": "nguyenvana@example.com",
  "soDienThoai": "0912345678",
  "diaChi": {
    "duong": "Lê Lợi",
    "quan": "Hoàn Kiếm",
    "thanhPho": "Hà Nội"
  }
}
\end{verbatim}

\subsubsection{Thu thập thông tin biểu mẫu từ website}

Khi khách hàng điền form trên website:
\begin{enumerate}
    \item Thông tin được gửi đến n8n dưới dạng JSON
    \item Mỗi mục thông tin là một cặp key-value
    \item Dữ liệu có thể lồng nhau (như địa chỉ ở trên)
\end{enumerate}

\subsection{Expressions: Công cụ làm việc với dữ liệu}

(bổ sung cho phần trên)

Expressions giống như công thức trong Excel, giúp bạn tính toán và xử lý dữ liệu.

\subsubsection{Ví dụ: Tính giá sau giảm giá}

Giả sử bạn có shop online và muốn tính giá sau khi giảm 10\%:

\begin{enumerate}
    \item Trong Set node, tạo trường ``giaSauGiamGia'':
    \begin{verbatim}
    {{ $json.giaBanDau * 0.9 }}
    \end{verbatim}

    \item Định dạng giá tiền:
    \begin{verbatim}
    {{ ($json.giaSauGiamGia).toFixed(0).toString()
    .replace(/\B(?=(\d{3})+(?!\d))/g, ".") }}
    \end{verbatim}
\end{enumerate}

% \begin{figure}[h]
%     \centering
%     \includegraphics[width=0.8\textwidth]{images/expressions-tinh-gia}
%     \caption{Expressions tính giá giảm 10\%}
%     \label{fig:expressions-gia}
% \end{figure}

\textbf{Ứng dụng thực tế khác:}
\begin{enumerate}
    \item Chuyển đổi định dạng ngày: \verb|{{ new Date($json.ngayDatHang).toLocaleDateString('vi-VN') }}|
    \item Ghép họ và tên: \verb|{{ $json.ho + ' ' + $json.ten }}|
    \item Kiểm tra điều kiện: \verb|{{ $json.tuoi >= 18 ? 'Người lớn' : 'Trẻ em' }}|
\end{enumerate}

\section{Storage Node (Data, Cache, Database)}

\subsection{Data Storage: Lưu trữ thông tin giữa các lần chạy}

\subsubsection{Ví dụ: Đếm số lượt truy cập website}

\begin{enumerate}
    \item Mỗi khi có người truy cập:
    \begin{itemize}
        \item Lấy số lượt hiện tại từ Data Storage
        \item Tăng thêm 1
        \item Lưu lại vào Data Storage
    \end{itemize}
\end{enumerate}

% \begin{figure}[h]
%     \centering
%     \includegraphics[width=0.8\textwidth]{images/workflow-dem-luot-truy-cap}
%     \caption{Workflow đếm lượt truy cập website}
%     \label{fig:dem-luot-truy-cap}
% \end{figure}

\textbf{Các bước thực hiện:}
\begin{enumerate}
    \item Thêm HTTP Trigger (kích hoạt khi có truy cập)
    \item Thêm Data Storage node đầu tiên:
    \begin{itemize}
        \item Operation: Get
        \item Key name: visitCount
    \end{itemize}
    \item Thêm Function node để tăng số:
    \begin{verbatim}
    // Nếu chưa có dữ liệu, bắt đầu từ 0
    const currentCount = items[0].json.data ?
    parseInt(items[0].json.data) : 0;
    
    // Tăng thêm 1
    return [{ json: { count: currentCount + 1 } }];
    \end{verbatim}
    \item Thêm Data Storage node thứ hai:
    \begin{itemize}
        \item Operation: Set
        \item Key name: visitCount
        \item Value: \verb|{{ $json.count }}|
    \end{itemize}
\end{enumerate}

\subsection{Cache Node: Lưu trữ tạm thời}

\subsubsection{Ví dụ: Lưu cache kết quả tỷ giá tiền tệ}

Thay vì gọi API tỷ giá mỗi phút, bạn có thể:
\begin{enumerate}
    \item Gọi API tỷ giá
    \item Lưu kết quả vào Cache với thời hạn 1 giờ
    \item Các request trong 1 giờ tiếp theo sẽ dùng dữ liệu cache
\end{enumerate}

% \begin{figure}[h]
%     \centering
%     \includegraphics[width=0.8\textwidth]{images/workflow-cache-ty-gia}
%     \caption{Workflow cache tỷ giá tiền tệ}
%     \label{fig:cache-ty-gia}
% \end{figure}

\textbf{Thiết lập workflow:}
\begin{enumerate}
    \item Thêm Schedule Trigger (chạy mỗi phút)
    \item Thêm Cache node:
    \begin{itemize}
        \item Operation: Get
        \item Key name: exchangeRates
    \end{itemize}
    \item Thêm node IF (kiểm tra cache có dữ liệu không)
    \item Nếu không có dữ liệu:
    \begin{itemize}
        \item Gọi API tỷ giá qua HTTP Request
        \item Lưu kết quả vào Cache:
        \begin{itemize}
            \item Operation: Set
            \item Key name: exchangeRates
            \item TTL: 3600 (1 giờ)
        \end{itemize}
    \end{itemize}
\end{enumerate}

\subsection{Database Node: Làm việc với cơ sở dữ liệu}

\subsubsection{Ví dụ: Hệ thống đặt hàng đơn giản}

\begin{enumerate}
    \item \textbf{Lưu đơn hàng mới:}
    \begin{itemize}
        \item Khách điền form $\rightarrow$ HTTP Trigger nhận dữ liệu
        \item Lưu thông tin vào database:
        \begin{itemize}
            \item Operation: Insert
            \item Table: donhang
            \item Columns \& Values: thông tin từ form
        \end{itemize}
    \end{itemize}
\end{enumerate}


\begin{enumerate}\setcounter{enumi}{1}
    \item \textbf{Cập nhật trạng thái đơn hàng:}
    \begin{itemize}
        \item Admin cập nhật $\rightarrow$ HTTP Trigger
        \item Cập nhật database:
        \begin{itemize}
            \item Operation: Update
            \item Table: donhang
            \item Set: \verb|{{ $json }}|
            \item Where: id = \verb|{{ $json.id }}|
        \end{itemize}
    \end{itemize}

    \item \textbf{Tạo báo cáo đơn hàng theo ngày:}
    \begin{itemize}
        \item Trigger chạy vào cuối ngày
        \item Truy vấn database:
        \begin{itemize}
            \item Operation: Select
            \item Table: donhang
            \item Where: ngayDatHang = today
        \end{itemize}
        \item Gửi email báo cáo tới quản lý
    \end{itemize}
\end{enumerate}

\section{HTTP Request \& API Basics}

\subsection{Tương tác với API: Lấy thông tin từ dịch vụ bên ngoài}

\subsubsection{Ví dụ: Bot thời tiết Telegram}

Khi người dùng nhắn tên thành phố, bot sẽ:
\begin{enumerate}
    \item Nhận tin nhắn từ Telegram
    \item Lấy thông tin thời tiết từ API
    \item Gửi thông tin thời tiết về cho người dùng
\end{enumerate}

\begin{figure}[htbp]
    \centering
    \includegraphics[width=0.8\textwidth]{Chap1-7/tele-weather.pdf}
    \caption{Workflow bot thời tiết Telegram}
\end{figure}

\textbf{Các bước thực hiện:}
\begin{enumerate}
    \item Thêm Telegram Trigger
    \item Thêm HTTP Request node:
    \begin{itemize}
        \item Method: GET
        \item URL: \verb|https://api.openweathermap.org/data/2.5/weather|
        \item Query Parameters:
        \begin{itemize}
            \item q: \verb|{{ $json.content }}| (tên thành phố từ người dùng)
            \item appid: YOUR\_API\_KEY
            \item units: metric
            \item lang: vi
        \end{itemize}
    \end{itemize}
    \item Thêm Function node để định dạng tin nhắn:
    \begin{verbatim}
    const weather = items[0].json;
    return [{
      json: {
        chatId: $node["Telegram Trigger"].json.message.chat.id,
        content: `Thời tiết tại ${weather.name}:\n` +
                 `Nhiệt độ: ${weather.main.temp}°C\n` +
                 `Độ ẩm: ${weather.main.humidity}%\n` +
                 ` Trạng thái: ${weather.weather[0].description}`
      }
    }];
    \end{verbatim}
    \item Thêm Telegram node để gửi tin nhắn lại cho người dùng
\end{enumerate}

\subsection{Xác thực API: Kết nối an toàn với dịch vụ bên ngoài}

\subsubsection{Ví dụ: Tự động đăng bài lên Facebook Page}

Để đăng bài lên Facebook, bạn cần:
\begin{enumerate}
    \item Lấy Access Token từ Facebook
    \item Sử dụng token trong API call
\end{enumerate}

Có hình
% \begin{figure}[h]
%     \centering
%     \includegraphics[width=0.8\textwidth]{images/workflow-dang-facebook}
%     \caption{Workflow đăng bài Facebook}
%     \label{fig:dang-facebook}
% \end{figure}

\textbf{Thiết lập workflow:}
\begin{enumerate}
    \item Thêm HTTP Trigger (khi có bài viết mới)
    \item Thêm HTTP Request node:
    \begin{itemize}
        \item Method: POST
        \item URL: \verb|https://graph.facebook.com/v16.0/YOUR_PAGE_ID/feed|
        \item Query Parameters:
        \begin{itemize}
            \item access\_token: YOUR\_ACCESS\_TOKEN
            \item message: \verb|{{ $json.content }}|
            \item link: \verb|{{ $json.link }}| (nếu có)
        \end{itemize}
    \end{itemize}
\end{enumerate}

\subsection{Tích hợp dịch vụ: Kết nối nhiều hệ thống}

\subsubsection{Ví dụ: Hệ thống CRM đơn giản}

Khi có khách hàng mới từ website:
\begin{enumerate}
    \item Lưu thông tin vào Google Sheets
    \item Tạo task theo dõi trong Trello
    \item Gửi email thông báo cho nhân viên
\end{enumerate}

Có hình
% \begin{figure}[h]
%     \centering
%     \includegraphics[width=0.8\textwidth]{images/workflow-crm-don-gian}
%     \caption{Workflow CRM đơn giản}
%     \label{fig:crm-don-gian}
% \end{figure}

\textbf{Các bước thực hiện:}
\begin{enumerate}
    \item Thêm Webhook node (nhận thông tin từ form)
    \item Thêm Google Sheets node:
    \begin{itemize}
        \item Operation: Append
        \item Spreadsheet: Danh sách khách hàng
        \item Range: Sheet1!A:E
    \end{itemize}
    \item Thêm Trello node:
    \begin{itemize}
        \item Operation: Create Card
        \item List: Khách hàng mới
        \item Title: \verb|Liên hệ {{ $json.hoTen }}|
        \item Description: \verb|{{ $json.email }} - {{ $json.soDienThoai }}|
    \end{itemize}
    \item Thêm Send Email node để thông báo cho nhân viên
\end{enumerate}

\subsection{Xử lý lỗi API: Đảm bảo workflow hoạt động ổn định}

\subsubsection{Ví dụ: Backup dữ liệu từ hệ thống cũ sang mới}

Khi chuyển dữ liệu, cần xử lý các trường hợp lỗi:

Có hình
% \begin{figure}[h]
%     \centering
%     \includegraphics[width=0.8\textwidth]{images/workflow-xu-ly-loi-api}
%     \caption{Workflow xử lý lỗi API}
% \end{figure}

\textbf{Thiết lập workflow:}
\begin{enumerate}
    \item Thêm Trigger (bắt đầu chuyển dữ liệu)
    \item Thêm HTTP Request node để lấy dữ liệu từ hệ thống cũ
    \item Thêm IF node để kiểm tra kết quả:
    \begin{itemize}
        \item Condition: \verb|{{ $json.statusCode }}| Equals \verb|200|
    \end{itemize}
    \item Nếu thành công:
    \begin{itemize}
        \item Thêm HTTP Request để gửi dữ liệu lên hệ thống mới
    \end{itemize}
    \item Nếu thất bại:
    \begin{itemize}
        \item Thêm Send Email node để thông báo lỗi
        \item Lưu lỗi vào log file
    \end{itemize}
\end{enumerate}